%    Sidebar about panic:
% 	panic is the kernel's last resort: the impossible has happened and the
% 	kernel does not know how to proceed.  In xv6, panic does ...
\chapter{Traps, interrupts, and drivers}
\label{CH:TRAP}

When running a process, a CPU executes the normal processor loop: read an
instruction, advance the program counter, execute the instruction, repeat.  But
there are events on which control from a user program must transfer back to the
kernel instead of executing the next instruction.  These events include a device
signaling that it wants attention, a user program doing something illegal (e.g.,
references a virtual address for which there is no page table entry), or a user
program asking the kernel for a service with a system call.  There are three
main challenges in handling these events: 1) the kernel must arrange that a
processor switches from user mode to kernel mode (and back); 2) the kernel and
devices must coordinate their parallel activities; and 3) the kernel must
understand the interface of the devices.  Addressing these 3 challenges requires
detailed understanding of hardware and careful programming, and can result in
opaque kernel code.  This chapter explains how xv6 addresses these three
challenges.

\section{Systems calls, traps, and interrupts}

There are three cases when control must be transferred from a user program to
the kernel. First, a system call: when a user program asks for an operating
system service, as we saw at the end of the last chapter.
Second, an
\textit{trap}\index{trap}:
when a program performs an illegal action. Examples of illegal actions include
divide by zero, attempt to access memory for a page-table entry that is not
present, and so on.  Third, an
\textit{interrupt}\index{interrupt}:
when a device generates a signal to indicate that
it needs attention from the operating system.  For example, a clock chip may
generate an interrupt every 100 msec to allow the kernel to implement
time sharing.  As another example, when the disk has read a block from
disk, it generates an interrupt to alert the operating system that the
block is ready to be retrieved.

The kernel handles all interrupts, rather than processes
handling them, because in most cases only the kernel has the
required privilege and state. For example, in order to time-slice
among processes in response the clock interrupts, the kernel
must be involved, if only to force uncooperative processes to
yield the processor.

In all three cases, the operating system design must arrange for the
following to happen.  The system must save the processor's registers for future
transparent resume.  The system must be set up for execution
in the kernel.  The system must chose a place for the kernel to start
executing. The kernel must be able to retrieve information about the
event, e.g., system call arguments.  It must all be done securely; the system
must maintain isolation of user processes and the kernel.

A word on terminology: this chapter uses the terms trap and interrupt
interchangeably, but it is important to remember that traps are caused
by the current process running on a processor (e.g., the process makes
a system call and as a result generates a trap), and interrupts are
caused by devices and may not be related to the currently running
process.  For example, a disk may generate an interrupt when it is
done retrieving a block for one process, but at the time of the
interrupt some other process may be running.  This property of
interrupts makes thinking about interrupts more difficult than
thinking about traps, because interrupts happen concurrently with
other activities.

\section{RISC-V Interrupts}

Devices on the RISC-V development board can generate interrupts, and
xv6 must set up the hardware to handle these interrupts.  Devices
usually interrupt in order to tell the kernel that some hardware event
has occured, such as I/O completion.  Interrupts are usually optional
in the sense that the kernel could instead periodically check (or
\textit{poll}\index{poll}) the device hardware to check for new
events.  Interrupts are preferable to polling if the events are
relatively rare, so that polling would waste CPU time.

Devices can generate interrupts
at any time.  There is hardware on the board to signal the CPU
when a device needs attention (e.g., the user has typed a character on
the keyboard). We must program the device to generate an interrupt, and
arrange that a CPU receives the interrupt. 

Let's look at the timer device and timer interrupts.  We would like
the timer hardware to generate an interrupt, say, 100 times per
second so that the kernel can track the passage of time and so the
kernel can time-slice among multiple running processes.  The choice of
100 times per second allows for decent interactive performance while
not swamping the processor with handling interrupts.

The development board on which xv6 is running has a \textit{Core Local
  Interruptor (CLINT)}\index{Core Local Interruptor (CLINT)}, which
can be programmed to generate timer interrupts.  The CLINT is a
memory-mapped device located at \lstinline{02000000} (see
Fig.~\ref{fig:xv6_layout}), and can thus be programmed with ordinary
load and store instructions.  xv6 programs the CLINT during start
\lineref{kernel/start.c:/CLINT/}: it reads the clock from address
\lstinline{CLINT_MTIME}, adds 10,000 cycles to it, and stores that
value in this core's \lstinline{CLINT_MTIMECMP(id)} timer register.
The hardware clock runs at 1Mhz and after 10,000 cycles (i.e., 10ms) the clock
value will be as large as the value in \lstinline{CLINT_MTIMECMP}, and
the CLINT will interrupt core \lstinline{id}.

\section{RISC-V protection and interrupt handling}

The RISC-V has 3 protection levels: machine mode (most privilege),
supervisor mode (also called kernel mode), and user mode (least
privilege).  Xv6 mainly uses 2 levels: supervisor and user. On an
interrupt or exception, the processor may have to switch from user
mode to kernel mode (e.g., when an interrupt arrives while user code
is running on the processor).

Taking interrupt in user mode is challenging.  At the time of the
interrupt, the user program is running with user-level page table and
it will enter the kernel with that page table, but in the kernel xv6
must run with the kernel page table. Furthermore, the kernel shouldn't
use the stack of the user process, because it may not be valid.  The
user process may be malicious or contain an error that causes the user
\texttt{sp} to contain an address that is not part of the process's
user memory.

To addresses these challenges, xv6 uses a
\textit{trampoline}\index{trampoline} to switch page tables.  It maps
a page code to enter and exit the kernel in every process's address
space and in the kernel address space.  This page is mapped at the
same address in each address space.  This page contains the code that
switches page tables, and because the page is mapped at the same
virtual address, the instruction right after the page table switch
will translate correctly.  The trampoline also switches from a user
stack to a valid kernel stack.

XXX switching off interrupts from user to kernel

XXX outline the high-level plan for interrups in each mode.

XXX scratch register

\section{Code: machine mode interrupts}

By default all interrupts trap into machine mode.  xv6 programs the
RISC-V to delegate all interrupts and exceptions to supervisor mode
\linerefs{kernel/start.c:/w_medeleg/,/w_mideleg/}.
Timer interrupts, however, must be received in machine
mode. Therefore, xv6 programs the hardware to receive interrupts in
machine mode: it stores into the \lstinline{mtvec} register the
address to jump to on an interrupt \lineref{kernel/start.c:/w_mtvec/}
and enables timer interrupts in the \lstinline{mie} register
\lineref{kernel/start.c:/w_mie/}.

When the CLINT interrupts the processor, the processor switches its program
counter to the value stored in lstinline{mtvec} register, which is
\lstinline{machinevec}.  \lstinline{machinevec} is defined in assembly
\lineref{kernel/kernelvec.S:/machinevec/}.  \lstinline{machinevec}
starts out by saving \lstinline{a0} in the \lstinline{mscratch} register
and loading into \lstinline{a0}.  xv6 stored earlier in the
\lstinline{mscratch} register \lineref{kernel/start.c:/mscratch/} the
address \lstinline{mscratch0}, which points to an area of 5
\lstinline{uint64}s.  Entries 4 and 5 contain the address of the
core's \lstinline{CLINT_MTIME} and the new value
\linerefs{kernel/start.c:/scratch.4/,/scratch.5/}.
\lstinline{machinevec} saves registers \lstinline{a1} through
\lstinline{a4} in the scratch area so that it can use them for its own
computation.  Then, it reprograms \lstinline{CLINT_MTIMECMP} to
generate an interrupt in 10ms and sets the interrupt bit in the kernel
\lstinline{sip} register.  Setting this bit will cause the kernel to
receive an interrupt after \lstinline{machinevec} completes.
\lstinline{machinevec} completes by restoring the registers it used
and returning from machine mode \lineref{kernel/kernelvec.S:/mret/} to
the mode where it came from when it was interrupted.

The code what was interrupted cannot tell it was interrupted: the
processor is in exactly the same state as before the interrupt,
because the \lstinline{machinevec} has restored all registers.  This
idea of saving the processor state to run some code, and then
restoring the processor state to resume the original code is a pattern
we will see a few more times.

\section{Code: kernel mode interrupts}

Handling interrupts in supervisor/kernel mode is similar to handling
interrupts in machine mode.  The main difference is that xv6 saves all
processor state because it may want to switch to running a different
process.  In machine mode, xv6 just reprograms the CLINT and then
resumes the interrupted code.

The details are as follows. During start up xv6 stores the address of
\lstinline{kernelvec} in the trap vector register for supervisor mode
(the \lstinline{stvec} register)
\lineref{kernel/trap.c:/trapinithart/}.  \lstinline{kernelvec} saves
all registers on the current stack
\linerefs{kernel/kernelvec.S:/sd ra/,/sd t6/}
and then calls the C function \lstinline{kerneltrap}
\lineref{kernel/kernelvec.S:/call/}.  \lstinline{kerneltrap}
\lineref{kernel/trap.c:/kerneltrap/} performs a few sanity checks
(e.g., it checks that it was indeed interrupted in kernel mode) and
then yield the core so that another process can run.  We will discuss
the implementation of \lstinline{yield} in Chapter~\ref{CH:SCHED}.

At some point later, the scheduler may decide to resume the process
that was interrupted by the timer interrupt. The process will return
from \lstinline{yield}, perhaps running on a different core now.  It
restores \lstinline{sepc} register and \lstinline{sstatus} register so
that \lstinline{sret} in \lineref{kernel/kernelvec.S:/sret/} will
resume the instruction that was interrupted.  \lstinline{kerneltrap}
returns and restores all the saved registers
\linerefs{kernel/kernelvec.S:/ld ra/,/ld t6/}, adjusts the stack
pointer, and resumes at \lstinline{sepc}.

\section{Code: user mode interrupts}

The function \lstinline{usertrapret}
\lineref{kernel/trap.c:/usertrapret/} sets up the trampoline to leave
and enter the kernel.  It starts out by disabling interrupts
\lineref{kernel/trap.c:/intr_off/}; xv6 is switching from taking
interrupts from \lstinline{kerneltrap} to \lstinline{usertrap} and
during this switch xv6 is not ready to take interrupts.

Next, \lstinline{usertrapret} sets the address where to enter the
kernel (by writing \lstinline{trampin} into \lstinline{stvec}) and it
sets up the values the trampoline needs when entering the kernel: the
kernel page table, the kernel stack of this process, the function to
call, and the hartid
\linerefs{kernel/trap.c:/kernel_satp.=.r_satp/,/r_tp/}.  We will see
how these values are used when entering the kernel.

Then, \lstinline{usertrapret} sets up values the trampoline needs to
exit the kernel and ``return'' to user space.  It will set the
previous mode in the \lstinline{sstatus} register to user mode (0) and
enable user interrupts in user mode, it will set the \lstinline{sepc}
register to the address the user program should be resumed at, and
will set the \lstinline{satp} register to this process's page table
\linerefs{kernel/trap.c:/x.=.r_sstatus/,/p->pagetable/}.
Finally, \lstinline{usertrapret} calls
the trampoline function, passing two arguments: the address of
\lstinline{trampout} and the value for \lstinline{satp}.

\lstinline{trampout} \lineref{kernel/trampoline.S:/trampout/} switches
the processor to use the user page table.  After this instruction, the
processor will translate virtual address with the new page table; for
example, the address of the next instruction will be translated with
the new page table.  This address will be translated in the same way
with the kernel page table, because the mapping is the same in both
page tables.

To execute the remaining instructions, \lstinline{trampout} needs a
register (\lstinline{a0}). But, it must also restore \lstinline{a0} to
the user process's value for \lstinline{a0}, which is stored in the
process's trap frame \lstinline{p->tf}.  The kernel stores the new
value for \lstinline{a0} in \lstinline{sscratch} and later restores
\lstinline{a0} it from there \lineref{kernel/trampoline.S:/csrrw a0/}.
Now it has \lstinline{a0} is availabe, it uses it to restore all other
processor registers from \lstinline{p->tf}
\linerefs{kernel/trampoline.S:/ld ra/,/ld t6/}.  Then, it restores
\lstinline{a0} by switching the values of \lstinline{sscratch} and
\lstinline{a0}. After this instruction, \lstinline{a0} contains the
user process's \lstinline{a0} and \lstinline{sscratch} contains
\lstinline{p->tf}.  Having \lstinline{p->tf} in \lstinline{sscratch}
will be convenient when entering: this is the address where xv6 will
save registers.  Finally, \lstinline{trampout} calls \lstinline{sret},
which returns to the previous mode (user mode) with interrupts enabled
and resuming at the address in \lstinline{sepc}.  We are out of the
kernel.

If a timer interrupt happens, the processor will switch to kernel mode
with the program counter set to the value in \lstinline{stvec} (i.e.,
\lstinline{trampin}) and \lstinline{sepc} containing the address of
the instruction that was interrupted.  \lstinline{trampin} first saves
\lstinline{a0} and loads \lstinline{p->tf} into \lstinline{a0}
\lineref{kernel/trampoline.S:/csrrw a0/}.  Next, it saves all other
processor registers into \lstinline{p->tf}
\linerefs{kernel/trampoline.S:/sd ra/,/sd t6/}.  Now all registers are
saved, \lstinline{trampin} can use \lstinline{t0} to save the original
value of \lstinline{a0}.  It switches to the kernel stack of the
process, because the user stack may not be valid.  It switches to the
kernel page table; again, this works out because trampoline page is
mapped identically in the user and kernel address space.  Finally, it
calls \lstinline{usertrap} (its address is in \lstinline{t0}).  We are
back in the kernel with a kernel page table and a (valid) kernel
stack, running C code.

\section{Code: C trap handler}

One good property of the code above is that it also works for
exceptions.  If the user process wants to call a system call or
executes an illegal instruction (e.g., dividing by zero), xv6 uses
\lstinline{trampin} to enter the kernel, and \lstinline{usertrap} will
be invoked in all three cases (i.e., for system calls, exceptions, and
interrupts).

The \lstinline{usertrap} function changes \lstinline{stvec} to the
address of \lstinline{kerneltrap} because xv6 is in kernel mode and
when xv6 enables interrupts in kernel mode, interrupts should go
through \lstinline{kerneltrap}; there is no need to switch page tables
and stacks. \lstinline{usertrap} next saves the \lstinline{sepc}
because that register contains the instruction where the process
should be resumed after completing the interrupt, exception, or system
call.  If the user process entered the kernel to execute a system
call, the register \lstinline{scause} contains 8.  In that case, xv6
adds 4 to \lstinline{p->tf->epc}, to skip the instruction that entered
the kernel (\lstinline{scall}) so that on return to user space the
process will resume at the instruction after \lstinline{scall}).
\lstinline{usertrap} then enables interrupts and calls
\lstinline{syscall}.

If an exception causes the kernel/user transition, then xv6 records
that the user process must be killed and will call \lstinline{exit}.
(We will look at how xv6 does this cleanup in Chapter~\ref{CH:SCHED}.)
Otherwise, the cause was an interrupt, which are dealt with by
\lstinline{devintr}\index{devintr@\lstinline{devintr}}.  For a timer
interrupt, \lstinline{devintr} just do two things: increment the ticks
variable \lineref{kernel/trap.c:/ticks\+\+/}, and call
\lstinline{wakeup}\index{wakeup@\lstinline{wakeup}}.  The latter, as
we will see in Chapter~\ref{CH:SCHED}, may make a process runnable.

After completing the system call or handling an interrupt, xv6 yields
the core to another process if the interrupt was a timer interrupt.
Otherwise, \lstinline{usertrap} returns back to user space using
\lstinline{usertrapret}; this section started with explaining
\lstinline{usertrapret}.  We have seen all the machinery to enter and
leave the kernel.
 
\section{Code: Enabling/disabling interrupts}

A processor can control if it wants to receive external and timer
interrupts through the \lstinline{SIE_SEIE}
\index{SIE_SEIE@\lstinline{SIE_SEIE}} flag and \lstinline{SIE_STIE}
\index{SIE_STIE@\lstinline{SIE_STIE}} flag in the \texttt{sie}
register, respectively.  The bootloader disables interrupts during
booting of the main cpu and the other other processors.  The scheduler on
each processor enables interrupts \lineref{kernel/proc.c:/intr_on/}.  To
control that certain code fragments are not interrupted, xv6 disables
interrupts during these code fragments.  For example,
\lstinline{usertrapret} above clears interrupts
\lineref{kernel/trap.c:/intr_off/}.

\section{Code: System calls}

Chapter~\ref{CH:FIRST} ended with 
\lstinline{initcode.S}\index{initcode.S@\lstinline{initcode.S}}
invoking a system call.
Let's look at that again
\lineref{user/initcode.S:/SYS_exec/}.
The process pushed the arguments
for an 
\lstinline{exec}\index{exec@\lstinline{exec}}
call on the process's stack, and put the
system call number in
\texttt{a7}.
The system call numbers match the entries in the syscalls array,
a table of function pointers
\lineref{kernel/syscall.c:/syscalls/}.
The \lstinline{ecall} instruction
switches the processor from user mode to kernel mode, and will
cause the kernel to call \lstinline{syscall}, as we saw above.

\lstinline{Syscall}\index{Syscall@\lstinline{Syscall}}
\lineref{kernel/syscall.c:/^syscall/} 
loads the system call number from the trap frame, which
contains the saved
\texttt{a7},
and indexes into the system call tables.
For the first system call, 
\texttt{a7}
contains the value 
\lstinline{SYS_exec}\index{SYS_exec@\lstinline{SYS_exec}}
\lineref{kernel/syscall.h:/SYS_exec/},
and
\lstinline{syscall}
will invoke the 
\lstinline{SYS_exec} 'th 
entry of the system call table, which corresponds to invoking
\lstinline{sys_exec}.

\lstinline{Syscall}
records the return value of the system call function in
\lstinline{p->tf->a0}.
When the system call returns to user space,
\lstinline{usertrapret}
will load the values
from
\lstinline{p->tf}\index{p->tf@\lstinline{p->tf}}
into the machine registers
and return to user space
using
\lstinline{sret}.

Thus, when 
\lstinline{exec}
returns, it will return the value
that the system call handler returned
\lineref{kernel/syscall.c:/a0 = syscalls/}.
System calls conventionally return negative numbers to indicate
errors, positive numbers for success.
If the system call number is invalid,
\lstinline{syscall}\index{syscall@\lstinline{syscall}}
prints an error and returns \-1.

\section{Code: System call arguments}

Later chapters will examine the implementation of
particular system calls.
This chapter is concerned with the mechanisms for system calls.
There is one bit of mechanism left: finding the system call arguments.
The helper functions
\lstinline{argint},
\lstinline{argaddr},
\lstinline{argptr},
\lstinline{argstr},
and
\lstinline{argfd}
retrieve the 
\textit{n} 'th 
system call
argument, as either an integer, pointer, a string, or a file descriptor.
\lstinline{argint}\index{argint@\lstinline{argint}}
and
\lstinline{argaddr}\index{argaddr@\lstinline{argaddr}}
use the function
\lstinline{fetcharg}
to locate the
\textit{n}'th 
argument. The C calling conventions specify that argument 0 is passed
through
\texttt{a0},
argument 1 through
\texttt{a1}, ...,
argument 6 through
\texttt{a6}.

\lstinline{argint} 
calls 
\lstinline{fetchint}\index{fetchint@\lstinline{fetchint}}
to read the value at that address from user memory and write it to
\lstinline{*ip}.  
\lstinline{fetchint}
cannot simply cast the address to a pointer, because the user and the
kernel don't share the same page table. Instead,
\lstinline{fetchint} calls \lstinline{copyin} to copy the address from the
process's address space to the kernel address space.  Before
calling \lstinline{copying}, \lstinline{fetchint}
verifies that the
pointer lies in the process's addres
space: it checks
that the address is below
\lstinline{p->sz}\index{p->sz@\lstinline{p->sz}}.

\lstinline{copyin}\index{copyin@\lstinline{copying}}
\lineref{kernel/vm.c:/^copyin/} copies \lstinline{len} bytes to
\lstinline{dst} from virtual address \lstinline{srcva} in the page
table \lstinline{pagetable}.  It walks the page table in software to
determine the physical address \lstinline{pa0} for \lstinline{srcva}.
Since the kernel maps virtual address one-to-one on physical
addresses and maps all physical addresses, the kernel can use
\lstinline{pa0} as a virtual address and copy the bytes at
\lstinline{pa0} using \lstinline{memmove}.

\lstinline{fetchaddr}\index{fetchaddr@\lstinline{fetchaddr}},
is like
\lstinline{fetchint},
but retrieves 64-bit value instead of a 32-bit int.

\lstinline{argptr}\index{argptr@\lstinline{argptr}}
fetches the
\textit{n}'th 
system call argument and checks that this argument is a valid
user-space pointer.

\lstinline{argstr}\index{argstr@\lstinline{argstr}} 
interprets the
\textit{n}'th 
argument as a pointer.  It ensures that the pointer points at a
NUL-terminated string and that the complete string is located below
the end of the user part of the address space.

Finally,
\lstinline{argfd}\index{argfd@\lstinline{argfd}}
\lineref{kernel/sysfile.c:/^argfd/}
uses
\lstinline{argint}
to retrieve a file descriptor number, checks if it is valid
file descriptor, and returns the corresponding
\lstinline{struct}
\lstinline{file}.

The system call implementations (for example, in sysproc.c and sysfile.c)
are typically wrappers: they decode the arguments using 
\lstinline{argint},
\lstinline{argaddr},
\lstinline{argptr}, 
and 
\lstinline{argstr}
and then call the real implementations.
In chapter~\ref{CH:MEM},
\lstinline{sys_exec}
uses these functions to get at its arguments.

\section{Drivers}

A
\textit{driver}\index{driver}
is the code in an operating system that manages a particular device:
it tells the device hardware to perform operations,
configures the device to generate interrupts when done,
and handles the resulting interrupts.
Driver code can be tricky to write
because a driver executes concurrently with the device that it manages.  In
addition, the driver must understand the device's interface (e.g., which I/O
ports do what), and that interface can be complex and poorly documented.

The disk driver provides a good example.  The disk driver copies data
from and back to the disk.  Disk hardware traditionally presents the data on the
disk as a numbered sequence of 512-byte 
\textit{blocks} 
\index{block}
(also called 
\textit{sectors}): 
\index{sector}
sector 0 is the first 512 bytes, sector 1 is the next, and so on. The block size
that an operating system uses for its file system maybe different than the
sector size that a disk uses, but typically the block size is a multiple of the
sector size.  Xv6's block size is identical to the disk's sector size.  To
represent a block xv6 has a structure
\lstinline{struct buf}
\lineref{kernel/buf.h:/^struct.buf/}.
The
data stored in this structure is often out of sync with the disk: it might have
not yet been read in from disk (the disk is working on it but hasn't returned
the sector's content yet), or it might have been updated but not yet written
out.  The driver must ensure that the rest of xv6 doesn't get confused when the
structure is out of sync with the disk.

\section{Code: Disk driver}

Xv6 has a disk driver for \textit{virtio}\index{virtio} disks. The
virtio standard is a standard for devices for kernels that run on a
virtual machine.  It defines the interactions between a guest kernel
and the virtual machine monitor for virtual devices emulating network
devices, disk devices, etc.  Xv6 runs on \texttt{qemu} and
\texttt{qemu} supports a virtio disk device for RISC-V platforms.  The
base development board that xv6 targets (the SiFive's HiFive) doesn't
provide a disk.

Xv6 represent file system blocks using
\lstinline{struct buf}\index{struct buf@\lstinline{struct buf}}
\lineref{kernel/buf.h:/^struct.buf/}.
\lstinline{BSIZE}
\lineref{kernel/fs.h:/BSIZE/}
is identical to the IDE's sector size and thus
each buffer represents the contents of one sector on a particular
disk device.  The
\lstinline{dev}
and
\lstinline{sector}
fields give the device and sector
number and the
\lstinline{data}
field is an in-memory copy of the disk sector.
Although the xv6 file system chooses
\lstinline{BSIZE}
to be identical to the IDE's sector size, the driver can handle
a
\lstinline{BSIZE}
that is a multiple of the sector size. Operating systems often use
bigger blocks than 512 bytes to obtain higher disk throughput.

The
\lstinline{flags}
track the relationship between memory and disk:
the
\lstinline{B_VALID}\index{B_VALID@\lstinline{B_VALID}}
flag means that
\lstinline{data}
has been read in, and
the 
\lstinline{B_DIRTY}\index{B_DIRTY@\lstinline{B_DIRTY}} 
flag means that
\lstinline{data}
needs to be written out.


The kernel initializes the disk driver at boot time by calling
\lstinline{virtio_disk_init}\index{virtio_disk_init@\lstinline{virtio_disk_init}}
\lineref{kernel/virtio\_disk.c:/^virtio_disk_init/}
from
\lstinline{main}\index{main@\lstinline{main}}
\lineref{kernel/main.c:/virtio_disk_init/}.
\lstinline{virtio_disk_init} negotiates the simplest features with the
disk and sets up one queues for disk requests.  The driver can have
\lstinline{NUM} outstanding requests to the disk.

Disk accesses typically take milliseconds, a long time for a
processor.  Therefore, xv6 lets another process run on the CPU after a
disk I/O request and arranges to receive an interrupt when the disk
operation has completed.
\lstinline{plic_Init}\index{plic_init@\lstinline{plic_init}} programs
the \textit{PLIC}\index{PLIC} to accept interrupts from the disk during boot.  It
enables interrupts for \lstinline{VIRTIO0_IRQ}
\lineref{kernel/plic.c:/^plicinit/}.  If the disk generates an
interrupt, \lstinline{devintr} \lineref{kernel/trap.c:/^devintr/} will
see that \lstinline{VIRTIO0_IRQ} was enabled and invoke
\lstinline{virtio_disk_intr}.

After this setup, the disk is not used again until the buffer cache calls
\lstinline{virtio_disk_rw}\index{virtio_disk_rww@\lstinline{virtio_disk_rw}},
which updates a locked buffer
as indicated by the flags.
If
\lstinline{B_DIRTY}\index{B_DIRTY@\lstinline{B_DIRTY}}
is set,
\lstinline{virtio_disk_rw}
writes the buffer
to the disk; if
\lstinline{B_VALID}\index{B_VALID@\lstinline{B_VALID}}
is not set,
\lstinline{virtio_disk_rw}\index{virtio_disk_rw@\lstinline{virtio_disk_rw}}
reads the buffer from the disk.
\lstinline{Virtio_Disk_Rw}
\lineref{kernel/virtio\_disk.c:/^virtio_disk_rw/}
adds the buffer
\lstinline{b}
to the end of the queue of outstanding requests (or wait if all
\lstinline{NUM} entries are in use).

Having added the request to the queue and started it if necessary,
\lstinline{virtio_disk_rw}
must wait for the result.  As discussed above,
polling does not make efficient use of the CPU.
Instead,
\lstinline{virtio_disk_rw}\index{virtio_disk_rw@\lstinline{virtio_disk_rw}}
yields the CPU for other processes by sleeping,
waiting for the interrupt handler to 
record in the buffer's flags that the operation is done
\linerefs{kernel/virtio\_disk.c:/while.*VALID/,/sleep/}.
While this process is sleeping,
xv6 will schedule other processes to keep the CPU busy.

Eventually, the disk will finish its operation and trigger an
interrupt, which will result in a call to
\lstinline{virtio_disk_intr}\index{virtio_disk_intr@\lstinline{virtio_disk_intr}}
to handle it
\lineref{kernel/trap.c:/virtio_disk_intr/}.
\lstinline{Virtio_Disk_Intr}
\lineref{kernel/virtio\_disk.c:/^virtio_disk_intr/}
consults the first buffer in the queue to find
out which operation was happening.
\lstinline{virtio_disk_intr}
sets 
\lstinline{B_VALID}\index{B_VALID@\lstinline{B_VALID}},
clears
\lstinline{B_DIRTY}\index{B_DIRTY@\lstinline{B_DIRTY}},
and wakes up any process sleeping on the buffer
\linerefs{kernel/virtio\_disk.c:/info\[id\]\.b/,/wakeup/}.

XXX DMA allows the device direct access to physical memory.
The driver gives the device the physical address of the buffer's data and
the device copies directly to or from main memory,
interrupting once the copy is complete.
DMA is faster and more efficient than programmed I/O
and is less taxing for the CPU's memory caches.

XXX The IDE driver routes interrupts statically to a particular processor.  Some
drivers configure the IO APIC
to route interrupts to multiple processors to spread out
the work of processing packets.
For example, a network driver might arrange to deliver interrupts
for packets of one network connection to the processor that is managing that
connection, while interrupts for packets of another connection are delivered to
another processor.  This routing can get quite sophisticated; for example, if
some network connections are short lived while others are long lived and the
operating system wants to keep all processors busy to achieve high throughput.


\section{Real world}

Supporting all the devices on a typical computer in its full glory is
much work, because there are many devices, the devices have many
features, and the protocol between device and driver can be complex.
In many operating systems, the drivers together account for more code
in the operating system than the core kernel.

Actual device drivers are far more complex than the disk driver in this chapter,
but the basic ideas are the same:
typically devices are slower than CPU, so the hardware uses
interrupts to notify the operating system of status changes.
Modern disk controllers typically
accept a 
\textit{batch}\index{batch} 
of disk requests at a time and even reorder
them to make most efficient use of the disk arm.
When disks were simpler, operating systems often reordered the
request queue themselves.

Many operating systems have drivers for solid-state disks because they
provide much faster access to data.  But, although a solid-state disk
works very differently from a traditional mechanical disk, both
devices provide block-based interfaces and reading/writing blocks on a
solid-state disk is still more expensive than reading/writing RAM.
The virtio disk makes no distinction between a mechanical disk and
solid-state disk.

Other hardware is surprisingly similar to disks: network device
buffers hold packets, audio device buffers hold sound samples,
graphics card buffers hold video data and command sequences.
High-bandwidth graphics cards, and network cards—often use DMA
as the virtio disk does.

Some drivers dynamically switch between polling and interrupts, because using
interrupts can be expensive, but using polling can introduce delay until the
driver processes an event.  For example, a network driver that receives a
burst of packets may switch from interrupts to polling since it knows that more
packets must be processed and it is less expensive to process them using polling.
Once no more packets need to be processed, the driver may switch back to
interrupts, so that it will be alerted immediately when a new packet arrives.

If a program reads a file, the data for that file is copied twice.  First, it
is copied from the disk to kernel memory by the driver, and then later it is
copied from kernel space to user space by the 
\lstinline{read}
system call.  If the program then sends the data over the network, 
the data is copied twice more: from user space to kernel space and from
kernel space to the network device.  To support applications for which 
efficiency is important (e.g., serving popular images on the Web), operating systems
use special code paths to avoid copies.  As one example,
in real-world operating systems, 
buffers typically match the hardware page size, so that
read-only copies can be mapped into a process's address space
using the paging hardware, without any copying.

\section{Exercises}

1. Set a breakpoint in
\lstinline{trap}
to catch the first timer interrupt. What values are on the stack at this
point?  Explain the output of x/37x
\texttt{\%rsp}
at that breakpoint with each value
labeled as to what it is (e.g., saved
\texttt{\%ebp}
for trap, trapframe.rip, scratch space, etc.).

3. Write a driver for a disk that supports the SATA standard (search for SATA on
the Web). Unlike IDE, SATA isn't obsolete.  Use SATA's tagged command queuing to
issue many commands to the disk so that the disk internally can reorder commands
to obtain high performance.

4. Add simple driver for an Ethernet card.
